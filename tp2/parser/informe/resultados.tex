\section{Procesamiento de textos en Ingl�s}

\subsection{Postagging Accuracy y las cinco dimensiones de error m�s frecuentes}

Para calcular esto primero convertimos los archivos de postagging en formato 
OpenNLP a formato BIO y luego utilizamos el programa para comparar postagged
que nos permit�a obtener la m�trica necesaria y las dimensiones de error m�s 
frecuentes.

Decidimos que no importaba el en que archivo se produciera la diferencia de 
etiqueta con el otro, es decir no importaba el orden. Por ejemplo que los errores 
NP NNP eran equivalentes a los NNP NP a la hora de sacar la estad�stica. 

\subsubsection{Genia}

Posttaging Accuracy: 0.866182504333326

\begin{center}
	\begin{tabular}{| c | c | c |}\hline
	Etiqueta & Cantidad & Incidencia\\\hline
	NN NNP   & 2661     & 0.43630103295622236\\
	JJ NN    & 1136     & 0.18626004262993934\\
	JJ VBN   & 197      & 0.0323003771110018\\
	NN CD    & 143      & 0.023446466633874407\\
	NNS VBZ  & 137      & 0.022462698803082472\\\hline
	Subtotal & 4274     &0.7007706181341203\\\hline
	\end{tabular}
\end{center}

\subsubsection{WsjSubset}

Posttaging Accuracy: 0.9715262680203475

\begin{center}
	\begin{tabular}{| c | c | c |}\hline
	Etiqueta & Cantidad & Incidencia\\\hline
	JJ NN    & 187      & 0.13862120088954782\\
	VBN VBD  & 129      & 0.09562638991845812\\
	IN RP    & 95       & 0.07042253521126761\\
	RB IN    & 87       & 0.06449221645663454\\
	NNP NNPS & 77       & 0.05707931801334322\\\hline
	Subtotal & 575      & 0.4262416604892513\\\hline
	\end{tabular}
\end{center}

\subsection{Chunking Precision y Recall para chunks verbales y nominales}

Para calcular estas dos m�tricas primero pasamos el archivo de chunking generado 
por OpenNLP al formato de BIO y luego utilizando un script hecho en java tomamos 
el archivo original y el procesado por opennlp para generar un archivo que en las 
tres primeras columnas ten�a la informaci�n original y en la �ltima la obtenida del 
procesamiento para luego poder utilizar el script conlleval.pl .

\subsubsection{Genia}

\texttt{./conlleval.pl -l < original/gen/genia.compare}

\begin{center}
	\begin{tabular}{| c | c | c | c |}
		\hline
			& Precision &  Recall  & F-Measure \\\hline
		ADJP    &   80.54\% &  70.92\% &  75.42 \\
		ADVP    &   74.36\% &  78.52\% &  76.38 \\
		CC      &    0.00\% &   0.00\% &   0.00 \\
		CD      &    0.00\% &   0.00\% &   0.00 \\
		CONJP   &  100.00\% &  18.03\% &  30.56 \\
		DT      &    0.00\% &   0.00\% &   0.00 \\
		FW      &    0.00\% &   0.00\% &   0.00 \\
		IN      &    0.00\% &   0.00\% &   0.00 \\
		JJ      &    0.00\% &   0.00\% &   0.00 \\
		JJ|RB   &    0.00\% &   0.00\% &   0.00 \\
		LS      &    0.00\% &   0.00\% &   0.00 \\
		LST     &    0.00\% &   0.00\% &   0.00 \\
		NN      &    0.00\% &   0.00\% &   0.00 \\
		NNS     &    0.00\% &   0.00\% &   0.00 \\
		NP      &   86.60\% &  82.57\% &  84.54 \\
		PP      &   92.32\% &  95.41\% &  93.84 \\
		PRT     &  100.00\% &  33.33\% &  50.00 \\
		RB      &    0.00\% &   0.00\% &   0.00 \\
		SBAR    &   92.93\% &  60.88\% &  73.57 \\
		VBN     &    0.00\% &   0.00\% &   0.00 \\
		VP      &   92.39\% &  92.91\% &  92.65 \\
 		DQE &    0.00\% &   0.00\% &   0.00 \\\hline
		Overall &   84.95\% &  86.34\% &  85.64 \\\hline
	\end{tabular}
\end{center}


De estas m�tricas las que nos interesan son VP (P:92.39\% - R:92.91\%) 
y NP (P:86.60\% - R:82.57\%). 


\subsubsection{WsjSubset}

\texttt{../../conlleval.pl -l < cmp/wsjsubset.compare}

\begin{center}
	\begin{tabular}{| c | c | c | c |}
		\hline
			& Precision &  Recall  & FF-Measure \\\hline
		''      &    0.00\% &   0.00\% &   0.00 \\
		ADJP    &   79.89\% &  67.12\% &  72.95 \\
		ADVP    &   80.79\% &  77.71\% &  79.22 \\
		CC      &    0.00\% &   0.00\% &   0.00 \\
		CD      &    0.00\% &   0.00\% &   0.00 \\
		CONJP   &   57.14\% &  44.44\% &  50.00 \\
		DT      &    0.00\% &   0.00\% &   0.00 \\
		IN      &    0.00\% &   0.00\% &   0.00 \\
		INTJ    &   50.00\% &  50.00\% &  50.00 \\
		JJ      &    0.00\% &   0.00\% &   0.00 \\
		JJR     &    0.00\% &   0.00\% &   0.00 \\
		LST     &    0.00\% &   0.00\% &   0.00 \\
		MD      &    0.00\% &   0.00\% &   0.00 \\
		NN      &    0.00\% &   0.00\% &   0.00 \\
		NNP     &    0.00\% &   0.00\% &   0.00 \\
		NNPS    &    0.00\% &   0.00\% &   0.00 \\
		NP      &   88.56\% &  90.06\% &  89.30 \\
		PP      &   94.23\% &  97.69\% &  95.93 \\
		PRP     &    0.00\% &   0.00\% &   0.00 \\
		PRT     &   75.00\% &  59.43\% &  66.32 \\
		RB      &    0.00\% &   0.00\% &   0.00 \\
		RBS     &    0.00\% &   0.00\% &   0.00 \\
		SBAR    &   87.84\% &  66.17\% &  75.48 \\
		TO      &    0.00\% &   0.00\% &   0.00 \\
		VB      &    0.00\% &   0.00\% &   0.00 \\
		VBD     &    0.00\% &   0.00\% &   0.00 \\
		VBN     &    0.00\% &   0.00\% &   0.00 \\
		VBZ     &    0.00\% &   0.00\% &   0.00 \\
		VP      &   93.06\% &  92.94\% &  93.00 \\
		``      &    0.00\% &   0.00\% &   0.00 \\\hline
		Overall &   84.88\% &  90.58\% &  87.63 \\\hline
	\end{tabular}
\end{center}

De estas m�tricas las que nos interesan son VP (P:93.06\% - R:92.94\%) 
y NP (P:88.56\% - R:90.06\%). 

\subsection{Resultados}

Se puede observar como mejoran los resultados tanto en chunking como postaggin para el wsjsubset en relaci�n al genia, que tiene t�rminos particulares de la biolog�a. 

Tambi�n se puede destacar como en las cinco dimensiones de error m�s frecuentes, se encuentran la mayor�a de lo errores, siendo de gran incidencia en ambos casos. 

En el caso de Genia es destacable que la principal y mayor causa de error es NN y NNP. Y en relaci�n ambos es relevante que alrededor del 0.15 se deben a JJ y NN.



\section{Resultados}

\subsection{Postagging Accuracy y las cinco dimensiones de error m�s frecuentes}

Para calcular esto primero convertimos los archivos de postagging en formato 
OpenNLP a formato BIO y luego utilizamos el programa para comparar postagged
que nos permit�a obtener la m�trica necesaria y las dimensiones de error m�s 
frecuentes.

Decidimos que no importaba el en que archivo se produciera la diferencia de 
etiqueta con el otro, es decir no importaba el orden. Por ejemplo que los errores 
NP NNP eran equivalentes a los NNP NP a la hora de sacar la estad�stica. 

\subsubsection{Genia}

Posttaging Accuracy: 0.866182504333326

\begin{center}
	\begin{tabular}{| c | c | c |}\hline
	Etiqueta & Cantidad & Incidencia\\\hline
	NN NNP   & 2661     & 0.43630103295622236\\
	JJ NN    & 1136     & 0.18626004262993934\\
	JJ VBN   & 197      & 0.0323003771110018\\
	NN CD    & 143      & 0.023446466633874407\\
	NNS VBZ  & 137      & 0.022462698803082472\\\hline
	Subtotal & 4274     &0.7007706181341203\\\hline
	\end{tabular}
\end{center}

\subsubsection{WsjSubset}

Posttaging Accuracy: 0.9715262680203475

\begin{center}
	\begin{tabular}{| c | c | c |}\hline
	Etiqueta & Cantidad & Incidencia\\\hline
	JJ NN    & 187      & 0.13862120088954782\\
	VBN VBD  & 129      & 0.09562638991845812\\
	IN RP    & 95       & 0.07042253521126761\\
	RB IN    & 87       & 0.06449221645663454\\
	NNP NNPS & 77       & 0.05707931801334322\\\hline
	Subtotal & 575      & 0.4262416604892513\\\hline
	\end{tabular}
\end{center}

\subsection{Chunking Precision y Recall para chunks verbales y nominales}

Para calcular estas dos m�tricas primero pasamos el archivo de chunking generado 
por OpenNLP al formato de BIO y luego utilizando un script hecho en java tomamos 
el archivo original y el procesado por opennlp para generar un archivo que en las 
tres primeras columnas ten�a la informaci�n original y en la �ltima la obtenida del 
procesamiento para luego poder utilizar el script conlleval.pl .

\subsubsection{Genia}

./conlleval.pl -l < original/gen/genia.compare 

\begin{center}
	\begin{tabular}{| c | c | c | c |}
		\hline
			& Precision &  Recall  & F-Measure \\\hline
		ADJP    &   80.54\% &  70.92\% &  75.42 \\
		ADVP    &   74.36\% &  78.52\% &  76.38 \\
		CC      &    0.00\% &   0.00\% &   0.00 \\
		CD      &    0.00\% &   0.00\% &   0.00 \\
		CONJP   &  100.00\% &  18.03\% &  30.56 \\
		DT      &    0.00\% &   0.00\% &   0.00 \\
		FW      &    0.00\% &   0.00\% &   0.00 \\
		IN      &    0.00\% &   0.00\% &   0.00 \\
		JJ      &    0.00\% &   0.00\% &   0.00 \\
		JJ|RB   &    0.00\% &   0.00\% &   0.00 \\
		LS      &    0.00\% &   0.00\% &   0.00 \\
		LST     &    0.00\% &   0.00\% &   0.00 \\
		NN      &    0.00\% &   0.00\% &   0.00 \\
		NNS     &    0.00\% &   0.00\% &   0.00 \\
		NP      &   86.60\% &  82.57\% &  84.54 \\
		PP      &   92.32\% &  95.41\% &  93.84 \\
		PRT     &  100.00\% &  33.33\% &  50.00 \\
		RB      &    0.00\% &   0.00\% &   0.00 \\
		SBAR    &   92.93\% &  60.88\% &  73.57 \\
		VBN     &    0.00\% &   0.00\% &   0.00 \\
		VP      &   92.39\% &  92.91\% &  92.65 \\
 		DQE &    0.00\% &   0.00\% &   0.00 \\\hline
		Overall &   84.95\% &  86.34\% &  85.64 \\\hline
	\end{tabular}
\end{center}


De estas m�tricas las que nos interesan son VP (P:92.39\% - R:92.91\%) 
y NP (P:86.60\% - R:82.57\%). 


\subsubsection{WsjSubset}

../../conlleval.pl -l < cmp/wsjsubset.compare 

\begin{center}
	\begin{tabular}{| c | c | c | c |}
		\hline
			& Precision &  Recall  & FF-Measure \\\hline
		''      &    0.00\% &   0.00\% &   0.00 \\
		ADJP    &   79.89\% &  67.12\% &  72.95 \\
		ADVP    &   80.79\% &  77.71\% &  79.22 \\
		CC      &    0.00\% &   0.00\% &   0.00 \\
		CD      &    0.00\% &   0.00\% &   0.00 \\
		CONJP   &   57.14\% &  44.44\% &  50.00 \\
		DT      &    0.00\% &   0.00\% &   0.00 \\
		IN      &    0.00\% &   0.00\% &   0.00 \\
		INTJ    &   50.00\% &  50.00\% &  50.00 \\
		JJ      &    0.00\% &   0.00\% &   0.00 \\
		JJR     &    0.00\% &   0.00\% &   0.00 \\
		LST     &    0.00\% &   0.00\% &   0.00 \\
		MD      &    0.00\% &   0.00\% &   0.00 \\
		NN      &    0.00\% &   0.00\% &   0.00 \\
		NNP     &    0.00\% &   0.00\% &   0.00 \\
		NNPS    &    0.00\% &   0.00\% &   0.00 \\
		NP      &   88.56\% &  90.06\% &  89.30 \\
		PP      &   94.23\% &  97.69\% &  95.93 \\
		PRP     &    0.00\% &   0.00\% &   0.00 \\
		PRT     &   75.00\% &  59.43\% &  66.32 \\
		RB      &    0.00\% &   0.00\% &   0.00 \\
		RBS     &    0.00\% &   0.00\% &   0.00 \\
		SBAR    &   87.84\% &  66.17\% &  75.48 \\
		TO      &    0.00\% &   0.00\% &   0.00 \\
		VB      &    0.00\% &   0.00\% &   0.00 \\
		VBD     &    0.00\% &   0.00\% &   0.00 \\
		VBN     &    0.00\% &   0.00\% &   0.00 \\
		VBZ     &    0.00\% &   0.00\% &   0.00 \\
		VP      &   93.06\% &  92.94\% &  93.00 \\
		``      &    0.00\% &   0.00\% &   0.00 \\\hline
		Overall &   84.88\% &  90.58\% &  87.63 \\\hline
	\end{tabular}
\end{center}

De estas m�tricas las que nos interesan son VP (P:93.06\% - R:92.94\%) 
y NP (P:88.56\% - R:90.06\%). 

\subsection{Evaluacion de un chunker en espa�ol}

Para esto utilizamos ChunkerTrainerMe de OpenNLP que nos permite generar un 
binario con nuestro modelo a partir de un archivo de entrenamiento. 

Luego testeamos este binario con el ChunkerEvaluator y los distintos archivos de testeo.

Probamos entrenarlo con A, AA, P y todos. Para esto utilizamos un script en java que nos permit�a 
unir todos los archivos de un directorio especificado en uno solo. 

El comando para generar el modelo es:

./bin/opennlp ChunkerTrainerME -encoding UTF-8 -lang es -data espaniol/train/X.train -model espaniol/bin/es-chunker.bin

Mientras que para evaluar los archivos de test:

./bin/opennlp ChunkerEvaluator -encoding UTF-8 -data espaniol/test/X.test -model espaniol/bin/es-chunker.bin

\subsubsection{Entrenado con AA}

\begin{enumerate}

\item Testeado con AA

\begin{itemize}
	\item Precision: 0.8170450806186246
	\item Recall: 0.7892561983471075
	\item F-Measure: 0.8029102667744543
\end{itemize}

\item Testeado con A

\begin{itemize}
	\item Precision: 0.771881461061337
	\item Recall: 0.7296416938110749
	\item F-Measure: 0.7501674480910918
\end{itemize}

\item Testeado con P

\begin{itemize}
	\item Precision: 0.8196829590488771
	\item Recall: 0.7810762614077416
	\item F-Measure: 0.7999140570446366
\end{itemize}

\end{enumerate}

\subsubsection{Entrenado con A}

\begin{enumerate}

\item Testeado con A

\begin{itemize}
	\item Precision: 0.7924784775713638
	\item Recall: 0.7596091205211727
	\item F-Measure: 0.7756957534094688
\end{itemize}

\item Testeado con P

\begin{itemize}
	\item Precision: 0.8016343893428859
	\item Recall: 0.751180111192699
	\item F-Measure: 0.775587566338135
\end{itemize}

\item Testeado con AA

\begin{itemize}
	\item Precision: 0.8012692050768203
	\item Recall: 0.7625556261919898
	\item F-Measure: 0.7814332247557003
\end{itemize}

\end{enumerate}

\subsubsection{Entrenado con P}

\begin{enumerate}

\item Testeado con P

\begin{itemize}
	\item Precision: 0.853230869141682
	\item Recall: 0.8269170250708067
	\item F-Measure: 0.8398678883443426
\end{itemize}

\item Testeado con A

\begin{itemize}
	\item Precision: 0.7801656592791583
	\item Recall: 0.756786102062975
	\item F-Measure: 0.7682980599647266
\end{itemize}

\item Testeado con AA

\begin{itemize}
	\item Precision: 0.8306464745155512
	\item Recall: 0.8107120152574698
	\item F-Measure: 0.8205581919086301
\end{itemize}


\end{enumerate}

\subsubsection{Entrenado con Todos}

\begin{enumerate}

	\item Testeado con P

	\begin{itemize}
		\item Precision: 0.8612941557740088
		\item Recall: 0.8363579146123991
		\item F-Measure: 0.8486428951569984
	\end{itemize}

	\item Testeado con A

	\begin{itemize}
		\item Precision: 0.8256840247131509
		\item Recall: 0.8125950054288816
		\item F-Measure: 0.8190872277552808
	\end{itemize}

	\item Testesado con AA

	\begin{itemize}
		\item Precision: 0.8492927979190376
		\item Recall: 0.8302606484424666
		\item F-Measure: 0.8396688901390341
	\end{itemize}

\end{enumerate}


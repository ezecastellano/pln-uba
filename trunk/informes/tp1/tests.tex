\section{Casos de Prueba}

Realizamos dos tipos de pruebas, tests hechos en C++ de las clases implementadas
al estilo unit testing, que se pueden ver en los archivos 
\texttt{../src/tests/test-automata.cpp} y \texttt{../src/tests/test-graph.cpp}.

Luego para validar la integridad y correctitud del programa \texttt{grep-line},
definimos un conjunto de expresiones regulares y la salida esperada dado un corpus
de texto. Las expresiones regulares fueron elegidas especificamente para abarcar
la mayor cantidad de casos posibles.

Las casos de prueba utilizados para validar la gramática fueron los siguientes:

\lstinputlisting{../src/tests/test-regexp.}

Con el siguiente corpus de texto:
\lstinputlisting{../src/tests/texto.}

Los resultados para estas expresiones regulares y estos textos se pueden encontrar 
en el directorio \texttt{tests/cases/}, donde el número de cada archivo se corresponde 
con el número de expresión regular del listado comenzando desde cero.

Para realizar las pruebas sobre la gramática lo que realizamos fue un archivo
de resultado esperado para cada expresión regular dado el corpus de texto, 
luego corrimos y utilizando la herramienta diff y un script en bash para hacer 
el chequeo más dinámico.

Todos los tests fueron correctos y el programa fue testeado en los aspectos más relevantes.

\bigskip

\section{Informaci\'on y requerimientos de software}

En esta secci\'on se indicar\'asn los requerimientos y las instrucciones necesarias para la ejecuci\'on del trabajo pr\'actico.

\subsection{Requerimientos}

El trabajo pr\'actico fue implementado en Java. Para poder ejecutar el programa es necesario tener instalado la Java Runtime Enviroment. 


\subsection{Ejecutar}

Para ejecutar el programa se deber estar posicionado desde la consola en en el directorio donde se encuentre el jar.

Ejecutar el siguiente comando: \texttt{java -jar tokenizador.jar archivo salida}

Donde salida puede ser A (archivo) o P (pantalla). 

En caso de elegir el tipo de salida archivo, este ser\'a dejado en el mismo directorio que el original con la extensi\'on .out. 

\subsection{Sintaxis}

Las siguientes son algunas de las abreviaturas de java para expresiones regulares comunes que fueron utilizadas en el desarrollo del trabajo pr�ctico:

Caracteres predefinidos:

\begin{itemize}
	 \item . Cualquier caracter. 
	 \item \textbackslash d	Un d�gito: [0-9]
	 \item \textbackslash D	Un no d�gito: [\^0-9]
	 \item \textbackslash s	Un caracter de espacio: [ \textbackslash t\textbackslash n\textbackslash x0B\textbackslash f\textbackslash r]
	 \item \textbackslash S	Un caracter que no sea espacio: [\^\textbackslash s]
	 \item \textbackslash w	Un caracter de palabra: [a-zA-Z\_0-9]
	 \item \textbackslash W	Un caracter no de palabra: [\^\textbackslash w]
\end{itemize}
 
Para m�s detalle consultar la documentaci�n de Java: \url{http://java.sun.com/developer/technicalArticles/releases/1.4regex/}.

\subsection{Archivos}

En este informe se entregan los siguientes archivos que corresponden a cada una de las tokenizaciones en sus diferentes formatos. El archivo de entrada, las dos salidas y el bueno cuyo formato varia. 

\begin{enumerate}
	\item original: Archivos de entrada y salida del corpus de entrenamiento. 
	\begin{itemize}
		\item \texttt{original.txt}
		\item \texttt{original.tok}
		\item \texttt{original.out}
		\item \texttt{original\_bueno.tok}		
	\end{itemize}
	\item prueba: Archivo de prueba que se encontraba en el sitio con su tokenizaci�n.
	\begin{itemize}
		\item \texttt{prueba.txt}
		\item \texttt{prueba.tok}
		\item \texttt{prueba.out}		
		\item \texttt{prueba\_bueno.out}		
	\end{itemize}
	\item espanol: Archivo correspodiente al texto seleccionado del espa�ol para las pruebas. 
	\begin{itemize}
		\item \texttt{espanol.txt}
		\item \texttt{espanol.tok}
		\item \texttt{espanol.out}		
		\item \texttt{espanol\_bueno.tok}		
	\end{itemize}
\end{enumerate}



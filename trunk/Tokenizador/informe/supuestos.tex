\section{Supuestos y decisiones}

Al realizar este tokenizador supusimos que los espacios eran un delimitador estricto en la tokenizaci\'on, as\'i como los saltos de l\'inea nos dividen en p\'arrafos siempre. Esto nos trajo un inconveniente en un token, el cual ser� detallado en la secci�n "Resultados". 

Para el an�lisis de algunas contracciones especiales utilizamos un diccionario de contracciones y supusimos que nuestro diccionario era bastante completo, de forma que los casos que no sean contemplados no representen un gran porcentaje de las contracciones.

Respecto a las abreviaturas decidimos que una palabra que empezaba con may�scula y terminaba en punto era considerada una abreviatura. Como no pod�amos sacar informaci�n del contexto debido a una tokenizaci�n inicial por espacios no ten�amos alternativas para distinguir un caso del otro, es por esto que optamos por considerarlas abreviaturas, ya que era lo que nos daba un mejor resultado en nuestro corpus de entrenamiento. 
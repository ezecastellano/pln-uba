\section{Conclusiones}

La realizaci\'on del trabajo pr\'actico fue interesante para comprender lo complejo del proceso. 

Fue interesante ver todo lo que nos permit� hacer una herramienta como OpenNLP respecto al procesamiento de textos y las posibilidades que existen de entrenarlo e ir evaluando con la misma herramienta. Si bien en este caso la documentaci�n era abundante y precisa, en muchas de las otras herramientas, esta era escasa.   

A su vez nos pareci� que fue necesario hacer demasiado trabajo extra para poder realizar los distintos procesos por el tema de los formatos y las peque�as diferencias en los tags del ingl�s. Ser�a interesante que se utilizar� una convenci�n general en el mundo del procesamiento de lenguajes para evitar todo este tipo de asuntos que no hacen a la soluci�n del problema. 

Respecto del procesamiento en ingl�s obtenemos como principal conclusi�n lo importante del corpus de entrenamiento respecto a las palabras particulares de cada ``�mbito''. En Genia la performance bajaba mucho porque la herramienta no se encontraba entrenada sobre textos de biolog�a y por lo tanto confund�a muchos nombres propios con sustantivos. 

En cuanto al espa�ol se siente la falta de trabajo respecto a este lenguaje, al menos a la hora de buscar herramientas ya entrenadas e informaci�n sobre los tags y modelos. Es comprensible que se trate de avanzar sobre el ingl�s por ser uno de los idiomas m�s �tilizado en las �reas que se encargan de investigar estos temas y tambi�n por ser m�s simple, pero consideramos que ser�a bueno que se fomente el procesamiento del lenguaje espa�ol.


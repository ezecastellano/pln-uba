\section{Casos de Prueba}

Para las pruebas fuimos realizando algunas casos iniciales de prueba para las expresiones regulares generadas que fueron ejecutadas.

Luego se implement\'o una salida etiquetada que permitida saber por que caso fue separado cada token, es por esto que la clase Token permit\'ia conservar su regla. 
Esta clase ten\'ia implementado el m\'etodo toString de la al estilo xml,  \textless regla\textgreater  token \textless /regla \textgreater y de

A su vez una vez realizado un avance en la tokenizaci\'on del texto, se tom\'o uno de los archivos procesados del estilo .tok y se modifico hasta lograr un archivo que fue considerado como bueno. 

Luego al procesar el archivo se comparaba utilizando el m\'etodo diff, cuales eran las diferencias entre el archivo procesado y el bueno, para ir pudiendo detectar cuales eran los detalles que faltaban corregir en el procesamiento del texto. Puede verse este archivo en el ap�ndice.

Por otro lado se realizaron pruebas con otro texto en ingl�s y con un texto en espa�ol con el objetivo de analizar cuan efectiva es la tokenizaci�n ante el cambio del corpus. 

\bigskip

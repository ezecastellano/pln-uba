\section{Resultados}

\subsection{Pruebas sobre el corpus original}

De la comparaci�n en entre el archivo "bueno" y la tokenizaci�n con reglas obtuvimos una diferencia que decidimos no resolver. Este problema se deb�a a la aparici�n de una abreviatura que pod�a a su vez pod�a ser considerada con una palabra que finalizaba con punto final. 

El caso no del todo satisfactorio era el siguiente, "1.4MB/25 sec." que nuestro tokenizador lo separaba de la separaba de la siguiente manera: ["1.4MB/25", "sec", "."] mientras que nosotros consider�bamos como bueno ["1.4MB/25", "sec."].  

Esto se debe a la noci\'on de abreviatura que utilizamos en las expresiones regulares, pues no hab\'ia ninguna que evitara que una palabra de tres caracteres que empiece con min\'uscula y finalice con un punto sea distinguida de un palabra de fin de oraci\'on. 

Para lograr salvar estos casos pensamos en la opci\'on de relevar cuales eran las abreviaturas m\'as conocidas del ingl\'es, pero no nos parec\'ia del todo correcto seguir agregando herramientas que no surgieran de las expresiones regulares. 

\subsection{Pruebas sobre otro corpus en ingl�s}

Para realizar una comparaci�n entre el archivo de control y nuestro tokenizador procesamos el texto que figura en el sitio y lo comparamos tambi�n utilizando el comando diff para ver en que puntos hab�a diferencia en nuestra tokenizaci�n. En el siguiente cuadro se pueden ver cuales fueron las diferencias. 

\begin{center}
	\begin{tabular}{| c | c |}
	\hline
   	prueba.out 		& prueba\_bueno.out 	\\
	\hline
  	GCN2			& GCN2(-/-) 	\\
  	( 				&				\\
	-				&				\\
	/				&				\\
	-				&				\\
	)				&				\\
	\hline
	\textless PP\textgreater	& \textless PP\textgreater		\\
					& \textless PP\textgreater		\\
	\hline
	(				&(active)-mTOR	\\
	active)-mTOR	&				\\
	\hline
	\end{tabular}
\end{center}
 
De la primer y la tercer diferencia podemos tomar como conclusi�n que la expresi�n regular que utilizamos para las palabras compuestas con barra o gui�n medio no capturaba todas las posibilidades, ya que solo ten�a en cuenta palabras compuestas con palabras y/o n�meros. A su vez si es algo m�s complejo y que lo que consideramos un token y tiene s�mbolos en los alguno de los bordes decidimos comenzar a separarlos y este es el motivo de nuestro resultado. Para solucionar este tema deber�amos generar una regular m�s compleja que la utilizada actualmente, pero que no permita cualquier combinaci�n (.+(-|/).+), pues si permitiese cualquier combinaci�n no nos permitir�a, entre otras cosas, sacar los par�ntesis de alrededor de estas palabras, que es algo que desear�amos que pase. Por ejemplo(GCN2(-/-)) quisieramos que fuera tokenizado de la siguiente manera ["(","GCN2(-/-)",")"].

De la segunda podemos detectar una ventaja de nuestra implementaci�n en la detecci�n de dobles salto de l�nea, que la implementaci�n correspondiente a la otra tokenizaci�n no la presenta. 

\subsection{Pruebas sobre un corpus en espa�ol}

Respecto al texto que fue seleccionado del espa�ol el principal problema hallado fueron las tildes, pues no las reconoc�a como un caracter de una palabra, motivo por el cual nos etiquetaba todo lo que tuviera tilde bajo la regla \textless raro\textgreater. 

Luego solo hubo una diferencia la tokenizaci�n semi-manual, \textless guion\textgreater [a-zA-Z]\textbackslash d\textless /guion\textgreater fue tokenizado como \textless simbolo\textgreater[\textless /simbolo\textgreater\textless raro\textgreater a-zA-Z]\textbackslash d\textless /raro\textgreater.

En el resto de la tokenizaci�n el proceso fue satisfactorio. De todas maneras la mayor�a de los problemas obtenidos fueron motivo de las tildes, por lo que fue etiquetado como algo an�malo, pero tokenizado correctamente. 

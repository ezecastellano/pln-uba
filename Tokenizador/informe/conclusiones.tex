\section{Conclusiones}

La realizaci\'on del trabajo pr\'actico fue interesante para aprender a utilizar las expresiones regulares en el lenguaje que normalmente utilizo para trabajar.

Nunca hab\'ia aprovechado el poder de las regexp en el \'ambito laboral ni acad�mico y es una herramienta muy \'util para muchas validaciones sint\'acticas. 

A su vez fue interesante ver que no es tan sencillo realizar algo que parec\'ia tan simple como un tokenizador, y que dependiendo del texto de entrenamiento que se utilice pueden variar mucho los resultados. Adem�s el tema de definir una expresi�n regular que capture un tipo de palabra, como por ejemplo las abreviaturas, no era tan sencillo, pues uno no sabe de donde puede provenir esta abreviatura y de esta forma tal vez no distinguirla de una oraci�n con punto final.

En definitiva es curioso ver que algo que uno a simple vista podr�a distinguir es tan dif�cil de lograr sin noci�n del contexto, ni del vocabulario posible. 

\section{Trabajo Futuro}

Con el objetivo de mejorar la tokenizaci�n proponemos mejorar las expresiones regulares correspondientes a las expresiones con barra y gui�n para poder capturar los tokens de la misma manera que otros tokenizadores.

A su vez ser�a conveniente mejorar la manera en que se distinguen las abreviaturas para poder capturar una mayor cantidad de casos.  

Otro tema a desarrollar es la distinci�n de contracciones especiales que actualmente utiliza un diccionario, poder distinguirlas a trav�s de reglas en vez de utilizar un diccionario, que nos restringe al conjunto de palabras que se encuentran en �l. 